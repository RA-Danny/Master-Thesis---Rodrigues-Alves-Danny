\chapter{Introduction}

Artificial intelligence is completely disrupting all kind of market. From self-drive cars to smart houses and IoT systems, everyone and everything is being connected and influenced by it. Businesses are not an exception and in today world, companies need to have a clear strategy about what to do with their data. From business processes to customer interaction, being able to generate useful information from data is primordial. This master thesis focuses on how companies can leverage artificial intelligence to make an intelligent and efficient use of their customer data.


% -------------------------------- Section: Outline of the project
\section{Outline of the project}

More and more companies are shifting from a product-centric view to a customer-centric focus. The objective is to create a one-to-one relationship with customers, to offer personalized products and services tailored to their needs and build successful customer relationships. To this end, Customer Relationship Management (CRM) systems are at the core of such business strategies. Companies store and organized all customer-related information into a CRM, from personal data to interaction with a sales agent for example. Disregarding the size of a business, more and more data is generated but few have the capacity to fully exploit it: the data is just stored in their system and access if needed. Having a dedicated resource to classify, analyze and generate insights from these data isn't an option. The new challenge faced by companies is to make an intelligent and efficient use of the huge customer data and integrate it into business processes.

Artificial Intelligence (AI) and Data analytics techniques are perfectly suited to exploit customer data. At the first lodge to support this new business need, CRM editors have embraced this market. For example Salesforce has \textit{Einstein}, Microsoft integrates \textit{Azure} solutions with Dynamics 365, SAP launched \textit{Leonardo} and Oracle develops some \textit{adaptive intelligent} apps. In a nutshell, all CRM editors claim to have some \textit{product} to combine CRM data with artificial intelligence techniques. CRM users are promised with a kind of smart assistance in a situation where they once found themselves alone ingesting too much information. 

This thesis studies how intelligent services can bring benefit to companies by being integrated with their CRM. In section \ref{sec:crm-ai}, current offers from the CRM market are analyzed. CRM editors are also evaluated in regards to their current and future AI's initiatives. Section \ref{sec:use-case} describes the creation of a predictive machine learning model based on CRM data. This model is also deployed and integrate directly inside an energy company's CRM.


% -------------------------------- Section: Context
\section{Context}
This master thesis is proposed by the CRM's division of ELCA Informatique SA. Founded in 1968 as \textit{Electro-Calcul}, ELCA is an independent Swiss IT company proposing a complete range of services in all IT areas like consulting, development, integration and operations. As a partner of Microsoft, the CRM teams at ELCA works with the Dynamics 365 CRM. Regarding the development of an artificial intelligence solution within a CRM, one of ELCA's client agreed to participate in this research and share its data.

The research detailed in section \ref{sec:crm-ai} is based on analyst reports, blog posts, knowledge based articles, and CRM's editors web pages. For the development of a machine learning solution outlined in section \ref{sec:use-case}, \textit{Jupyter Notebooks} have been used to work with the data. The development has been done in \textit{Python}, based mainly on \textit{Pandas}, \textit{Scikit-learn} and \textit{Keras} libraries.

This project has been done in the context of an EPFL master thesis between February to August 2018, supervised by the \textit{Distributed Information System laboratory}.