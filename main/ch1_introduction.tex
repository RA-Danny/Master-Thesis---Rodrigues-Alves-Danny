\chapter{Introduction}

Artificial intelligence is completely disrupting all kinds of markets. From self-driving cars to smart houses and Intern of Things (IoT) systems, everyone and everything is being connected and influenced by it. Companies are no exception and in today's world they need a clear strategy on what to do with their data. From business processes to customer interaction, being able to generate insights from data is primordial. Previously, the main goal regarding data was to acquire as many as possible and then build a clear vision about the customer.  Today, the objective is to use these amounts of data to guide companies. 


% -------------------------------- Section: Outline of the project
\section{Outline of the project}

More and more companies are shifting from a product-centric view to a customer-centric focus. The objective is to create one-to-one relationships with customers, to offer personalized products, to offer services tailored to their unique needs, and to build successful customer relationships. Customer Relationship Management (CRM) systems are at the core of such business strategies. Companies store and organize all their customer-related information into a CRM, such as a customer personal data or the customer interactions with a sales agent. Disregarding the size of a business, more and more data is generated but few have the capacity to fully exploit it: the data is just stored in their system and access if needed. Having a dedicated resource to classify, analyze, and generate insights from these data is not an option. The new challenge faced by companies is to make an intelligent and efficient use of the huge customer data, to generate new insights from it, and integrate them into business processes.

Artificial Intelligence (AI) and data analytics techniques are perfectly suited to exploit customer data. CRM's editors understand this new need of the business and invest in the market. For example Salesforce has \textit{Einstein}, Microsoft integrates \textit{Azure} solutions with Dynamics 365, SAP launched \textit{Leonardo}, and Oracle develops \textit{adaptive intelligent} apps. In a nutshell, all CRM's editors claim to have some \textit{intelligent product} to combine data with artificial intelligence techniques. Users are promised with a kind of smart assistance in a situation where they once found themselves alone ingesting too much information. 

This master thesis focuses on how companies can leverage artificial intelligence to make an intelligent and efficient use of their customer data. In section \ref{sec:crm-ai}, current offers from the market are analyzed. CRM's editors are compared in regards to their current and future AI's initiatives. Section \ref{chapter:use-case} describes the creation of a predictive machine learning model based on CRM data. This section outlines how a company can use its customer data to predict the date of a customer's next order. The model created is then deployed and integrated directly into the company's system, such that the results can be combined with current processes.

% -------------------------------- Section: Context
\section{Context}
The subject of this study is proposed by the CRM's division of ELCA Informatique SA. Founded in 1968 as \textit{Electro-Calcul}, ELCA is an independent Swiss company proposing a complete range of services in all Information Technology (IT) areas such as consulting, development, integration and operations. As a Microsoft's partner, the CRM team at ELCA works with \textit{Dynamics 365}, Microsoft's customer relationship management solution.
For the development of an artificial intelligence solution within a CRM reported in section \ref{chapter:use-case}, one of ELCA's client participates in this research and shares its data.

This project is done in the context of a Master thesis between February to August 2018, supervised by the \textit{Distributed Information System laboratory} (LSIR) at the École Polytechnique Fédérale de Lausanne (EPFL).