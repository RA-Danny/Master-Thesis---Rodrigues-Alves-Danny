\chapter{Conclusion}

\vspace{-0.5cm}
\epigraph{It’s difficult to make predictions, especially about the future}{\textit{Danish proverb}}
\vspace{0.5cm}

This master thesis is composed of two parts: a research about current offerings in the CRM AI field and the development of a machine learning model from and for a CRM.

Section \ref{sec:crm-ai} illustrates that CRM editors have entered the artificial intelligence market and none of them wants to miss this new segment. Deploying out-of-the-box features that are integrated inside the CRM is the current strategy. It enables users to experiment with AI through the provided features, features that are pretty much the same for every editor. This is clearly the first steps to augment the CRM with AI. A second step would be to enable users to create and integrate their own features, built to fit their unique needs. In terms of artificial intelligence techniques, predictive models based on historical data are the most used until now. The CRM AI field in full development, some features will be drop while some will be added as the use cases are defined by companies using it every day.

One example of a custom AI feature tailored to a business is outlined in section \ref{chapter:use-case}. The predictions made by the model are integrated directly into the CRM with just two new variables. This enables users to use machine learning outputs directly, in the same environment and with the same tools they are aware and used to. The architecture to bring predictions into the CRM automatically updates the model's results. This feature is autonomous, intelligent, and created to help CRM's users without requiring them to perform a task.

Regarding the neural network used for this predictions, its F1-score on testing and validation datasets are satisfactory. This task is not trivial, as the patterns for customer's orders do not always follow a certain logic. In general, predicting human behavior is a difficult task with several parameters that must be taken into account. Others features or even other models could produce better predictions. Similar to the LSTM network, a model that can take into account the entire history of a customer might be interesting.

This thesis demonstrates that artificial intelligence will be more and more present when dealing with customer relationship management. It will guide CRM's users on how to personalize their interactions with their customers and build a solid long-term relationship.