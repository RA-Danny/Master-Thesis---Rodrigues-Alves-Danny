\chapter{CRM AI: An use case} \label{chapter:use-case}

One of the goals of this project is to integrate an artificial intelligence solution within a CRM. After the analysis of \textit{CRM AI}, it has been decided to develop a predictive machine learning model based on CRM data. This part has been developed in collaboration with one of ELCA's client, hereafter referenced as \textit{Company A} \footnote{Client name and business subject to confidentiality.}.

This chapter \ref{chapter:use-case} details this machine learning project. Sections \ref{sec:use-case} and \ref{sec:ml-metrics} explains the problem \textit{Company A} is currently facing and how a machine learning solution might solve it.The data to feed the machine learning comed from Company A's CRM, as explained in section \ref{sec:crm-data}. Section \ref{sec:ml-experimentation} details the machine learning experimentation and models used. Once the model has been trained and tested, it was integrate within Company A's CRM, as outlined in section \ref{sec:crm-deployment}. Finally, sections \ref{sec:use-case-further-work} and \ref{sec:use-case-conclusion} conclude this part with a reflection on the entire initiative.

% -------------------------------- Section: Client's use case
\section{Client's use case} \label{sec:use-case}
\textit{Company A} is a global energy enterprise active overall Switzerland. It offers several products and services and one of those is subject to a very tense market, where the characteristics of the product are the same for all companies and the price is the only difference between competitors. Therefore, several specificity of this product must be taken into account:
\begin{itemize}
\item This product is classified as a safety need in Maslow's hierarchy of needs\cite{wiki:Maslow's_hierarchy_of_needs}, meaning that people don't buy it because the \textit{like} it but rather because they \textit{need} it. 
\item Due to its classification in Marslow's hierarchy, the product is usually stored in high quantities. Typical customer will make a large order, fill their supplies and once fully consumed, buy back the product, again in large quantities. Therefore, the product is not buy on a daily basis, more on an annual-basis. This complicates the job of Company A to build customer loyalty.
\item As product's characteristics are the same for all, the price is the only variable that companies can \textit{play} with. Nevertheless, a part of the price is subject to market variation, from which companies will adapt their margins. As all stock market, the prices can vary, even marginally, each day.
\end{itemize}

 Based on these particularities of the product, building a solid customer relationship is difficult for \textit{Company A}. Customers only need to make one order per year and their are usually not subject to an immediate need. Therefore, they can take the time to compare the price offered by all suppliers in the market and place an order accordingly. This facts explain why \textit{Company A} is often dealing with \textit{one-time customers} and experiences a high customer turnover \textbf{[ADD DATA SUPPORING THAT CLAIM]}.
 
 
 To counter this customer churn, \textit{Company A} is building \textit{customer recovery} plans to get back lost clients as well as interactions with other products and services to build an ecosystem of offers, but they want something uniquely targeting their key-product. Currently, Company A marketing team are contacting clients with annual newsletters. Those newsletters are send to each client every year around the same period. The company wants to strengthen this marketing process by contacting clients at the perfect time, before they start to search for offers from the competition. This will enable Company A to retain clients and ultimately build customer loyalty.
 
 
% -------------------------------- Section: Project outline
\section{Formalize machine learning problem} \label{sec:ml-metrics}

As defined in section \ref{sec:use-case}, the goal of this initiative is to build a machine learning model that predicts the time of a customer's next order. This problem can be formalize as a regression problem, for which the machine learning model will output the date of next order: at date $d$, the model will predict $i$, the number of days until next order. So $d+i$ will give the precise data of customer next order. Variants are to define $i$ as the number of weeks or months until next order. The date will be less precise but the predictions might turn out to be more satisfying. Another possibility is to formalize it as a classification task and output a binary value to the question : \textit{"Will this customer make an order in the coming day/week/month ?"}. After discussion with \textit{Company A}, it has been decided to define the output of the model as \textit{the number of months until customer next order}. In details, it has been decided first to work with regression models to have multiple ranges of outputs. Then, for the output's granularity (day, week or month), predicting the number of month until next order \textit{should} give stronger results. It will allow \textit{Company A} to asses how well a regression techniques are suited for this problem and, in a second phase, trying to have more precise predictions, with an output at the week level for example.

From the machine learning output, Company A wants to reach customers before they make an order. The process is planned as follow:
\begin{enumerate}
    \item At the beginning of each month, compute the predictions for all active clients\footnote{An client is flagged as inactive if no more business can be made with him/her (moving abroad or death for example)}.
    \item Retrieve all clients with a predictions smaller than $2$.
    \item Get in touch with these clients if and only if they have not been contacted in the past two months.
\end{enumerate}
The last step of this process is very important, as it will ensure that the company won't reach the same client twice.


Now that the machine learning goal and usage have been specified, the metrics for its success must be defined. As stated above, this is a regression model, where the output will be a real number. Usual metrics for regression problems like Mean Absolute Error (MAE) or Root Mean Squared Error (RMSE) are not well suited for this project and for the usage of the predictions by \textit{Company A}. As specified above, the company plans to get in touch with a client only if the predictions is below $2$. Therefore, if the output of a model is equal to $7.0$ months or $4.35$, it ends up being the same for \textit{Company A}. This explains why RMSE or MAE cannot be used to evaluate machine learning models performance. As regression metrics are not applicable, the models will be assess with custom metrics inspired by classification problems: precision, recall and F1 score. 

To transform the regression problem into a classification one, model's predictions will be classify into one of the following four classes: \texttt{0, 1, 2 or 3+}. The class \texttt{0} is meant for predictions between 0 and 1 (excluded) - clients that should make an order in the current month. Same idea for classes \texttt{1} and \texttt{2}. The class \texttt{3+} regroups all predictions with an output bigger or equal to 3 - client next order should occur in three or more months.

\begin{adjustwidth}{-1cm}{}
    \begin{minipage}[b]{0.60\linewidth}
    \resizebox{1.0\columnwidth}{!}{
        \begin{tabular}[t]{c|c|
                >{\columncolor[HTML]{EFEFEF}}c|c|
                >{\columncolor[HTML]{EFEFEF}}c }
                Customer & y\_true & y\_true class & y\_pred & y\_pred class \\ \hline
                A        & 2       & 2             & 3.32    & 3             \\
                B        & 0       & 0             & 0.4     & 0             \\
                C        & 0       & 0             & 0.1     & 0             \\
                D        & 5       & 3+            & 2.9     & 2            \\
                E        & 3       & 3+            & 0.8     & 0             \\
                F        & 1       & 1             & 1.6     & 1             \\
                G        & 1       & 1             & 0.8     & 0             \\
                H        & 2       & 2             & 1.4     & 1             \\
                I        & 9       & 3+            & 15.2    & 3+            \\
                J        & 0       & 0             & 2.5     & 2            
        \end{tabular}
        }
        \captionof{table}{Example of output truth and predictions}
        \label{table:example_truth_preds} 
    \end{minipage}
    \hspace{1.0cm}
    \begin{minipage}[b]{0.45\linewidth}
        \offinterlineskip
        \moveright 1cm \hbox{\raisebox{1.8cm}[10pt][10pt]{\rotatebox[origin=c]{90}{\parbox[c][0pt][c]{15cm}{True class\\[50pt]}}}\par}
        \hspace*{1cm}\MyHBoxT[\dimexpr5.1cm]{Predicted class}\vspace*{-0.4cm}
        \hspace*{1cm}\MyHBox{0}\MyHBox{1}\MyHBox{2}\MyHBox{3+}\vspace*{-0.2cm}
        \MyTBox{0}{2}{0}{1}{0}
        \MyTBox{1}{1}{1}{0}{0}
        \MyTBox{2}{0}{1}{0}{1}
        \MyTBox{3+}{1}{0}{1}{1}
        \captionof{table}{Confusion matrix for Table 3.1}
        \label{table:example_confusion_matrix} 
    \end{minipage}
\end{adjustwidth}


Based on this classes assignment, a confusion matrix can be generated. Then, custom metrics inspired by precision, recall and F1-score are computed:
\begin{itemize}
    \item \textbf{Precision}: Of all clients that make an order in the current month, how many where predicted with class 0 or 1? This metric computes the percentage of true orders catch by the model. 
    $$ Precision = \frac{t_0\_p_0 + t_0\_p_1}{t_0\_p_0 + t_0\_p_1 + t_0\_p_2 + t_0\_p_{3+}} $$
    
    For the example in table 3.2, the precision is equal to $\frac{2+0}{2+0+1+0} = 0.667$
    
    \item \textbf{Recall}: From all clients that the model has predicted an order for the current month, how many did actually made an order in the current or coming month? The goal with this metric is to assert that the model is not always predicting 0, which will give a precision score of 100\%, but will make \textit{Company A} contact all clients. 
    $$ Recall = \frac{t_0\_p_0 + t_1\_p_0}{t_0\_p_0 + t_1\_p_0 + t_2\_p_0 + t_{3+}\_p_0} $$
    
    For the example in table 3.2, the recall is equal to $\frac{2+1}{2+1+0+1} = 0.75$
    
    \item \textbf{F1-score}: Same as the F1-score for classification tasks. 
    $$ \fscore = \frac{2*Precision*Recall}{Precision+Recall} $$

    For the example in table 3.2, the F1-score is equal to $\frac{2*0.667*0.75}{0.667+0.75} = 0.71$
\end{itemize}

In the formulas above, $t_i\_p_j$ correspond to the sum of true orders occurring in \texttt{i} months and predicted to occur in \texttt{j} months. For example $t_0\_p_0$ corresponds to the count of orders occurring in the current month and predicted as such (top left case in the confusion matrix).

The metric \textbf{F1-score} will be the benchmark metric for this project.

% -------------------------------- Section: Data
\section{Data from CRM} \label{sec:crm-data}

Now that the machine learning parameters have been determined, the data to use with the model must be detailed. Data comes from \textit{Company A}'s CRM, a Dynamics 365 Online system. Thanks to the Web Api offered by Dynamics 365, all entities of the CRM have been retrieved and analysis. For the analysis of data and machine learning experiments, the development has been done within \textit{Jupyter Notebooks} with a \texttt{Python 3} kernel, mainly relying on the \textit{Pandas} package.

\subsection{Accounts, Clients and Orders}
From the 291 entities present in Company A's CRM, only a few are useful for this project: \texttt{Orders}, \texttt{Accounts}, \texttt{Contacts} and \texttt{Buildings}. The relations between those entities are straightforward: An order is linked to an account, each account is linked to at least one contact, the \textit{primary contact}. The contact entity models a real person, where the \texttt{Account} entity model a customer for Company A, either a private person or a company. \texttt{Contacts} can then be linked to one or more instance of the \texttt{Building} entity.

\refstepcounter{table}\label{caca}
{
\hskip-10.0cm
    \begin{table}[]
        \caption[DANNY The KING]{}
        \label{tab:my_label}
    \end{table}
}

\subsection{External data}

\subsection{Shape data for Machine Learning }
data from crm
weather
price
one entry per month



% -------------------------------- Section: Customer Jounrey + Feature engineering
\section{Feature engineering} \label{sec:ml-features}

feature engineering
Customer journey




% -------------------------------- Section: Machine Learning Experimentation
\section{Machine Learning Experimentation} \label{sec:ml-experimentation}
\lipsum[1]

\subsection{Linear models}
\lipsum[2]

\subsection{Neural Networks}
\lipsum[3]

\subsection{Best model analysis}
\lipsum[3]


% -------------------------------- Section: Deployment
\section{Deployment} \label{sec:crm-deployment}
How to use machine learning predictions within a Dynamics 365 CRM, based on Azure products

\subsection{Azure Architecture}
\lipsum[2]

\subsection{Integration with Dynamics 365}
\lipsum[3]


% -------------------------------- Section: Further Work
\section{Further Work} \label{sec:use-case-further-work}
\lipsum[1]

% -------------------------------- Section: Conclusion
\section{Conclusion} \label{sec:use-case-conclusion}
\lipsum[1]