\chapter{Use case: Next order prediction} \label{chapter:use-case}

One of the goals of this project is to integrate an artificial intelligence solution within a CRM. After the analysis reported in chapter \textit{CRM AI}, it has been decided to develop a predictive machine learning model based on CRM data. This part if the thesis has been developed in collaboration with one of ELCA's client, hereafter referenced as \textit{Contoso} \footnote{Client's name and business subject to confidentiality.}.

This chapter details the entire machine learning project. Sections \ref{sec:use-case} and \ref{sec:ml-metrics} explain the business problem \textit{Contoso} is currently facing and how a machine learning solution might solve it. The data contained in Contoso's CRM is explained in section \ref{sec:crm-data} and section \ref{sec:ml-experimentation} details the machine learning experimentations based on this data. The deployment and integration with Contoso's CRM of the model created are outlined in section \ref{sec:crm-deployment}. Finally, sections \ref{sec:use-case-further-work} and \ref{sec:use-case-conclusion} conclude the chapter with a reflection on the entire initiative.

% -------------------------------- Section: Client's use case
\section{Client's use case} \label{sec:use-case}
\textit{Contoso} is an energy-focus enterprise active overall Switzerland offering several products and services. One of their product is subject to a very tense market, where its characteristics are the same among all competitors in the market with the price as the only differentiator. When dealing with this product, several of its specificities must be taken into account:
\begin{itemize}
\item The product is classified as a \textit{safety need} in Maslow's \textit{hierarchy of }needs\cite{wiki:Maslow's_hierarchy_of_needs}, meaning that people don't buy it because the \textit{like} it but because they \textit{need} it. 
\item Due to its classification in Maslow's hierarchy, the product is usually stored in high quantities. The typical customer will make a large order, fill its supplies, consume the product and once fully consumed, the product is bought again in large quantities. Therefore, the product is not bought on a daily basis but rather on an annual basis. This is a real difficulty for Contoso while trying to build customer loyalty.
\item Since the product's characteristics are the same for the entire market, the price is the only variable that companies can \textit{play} with. Nevertheless, part of the price is subject to global market variation, from which companies will adapt their margins. As all stock market, the prices can vary, even marginally, each day.
\end{itemize}

 Based on these particularities, building a solid customer relationship is difficult for Contoso: Customers only need to make one order per year in average and they are usually not subject to an immediate need. Therefore, they can take the time to compare the price offered by all suppliers in the market and make an order accordingly. This explains why Contoso is facing a high customer turnover and often dealing with \textit{one-time customers}, as shown by figure \ref{fig-annex:orders_per_accounts}.
 
 
 To dwindle customer churn and get lost clients back, \textit{Contoso} is building \textit{customer recovery} plans and reinforcing interactions with other products and services to form a complete ecosystem, but they want something uniquely targeting their key-product. Currently, Contoso's marketing team are contacting clients with annual newsletters. Those newsletters are sent to each client every year around the same period. The company wants to strengthen this marketing process by reaching out to clients at the perfect time, juste before they start to search for offers from the competition. This will enable Contoso to retain clients and ultimately build customer loyalty.
 
 
% -------------------------------- Section: Project outline
\section{Formalize machine learning problem} \label{sec:ml-metrics}


\subsection{Problem definition}
As defined in the previous section, the goal of this initiative is to build a machine learning model that predicts the time of a customer's next order. This problem can be formalized as a regression problem, for which the machine learning model will output the date of next order: at date $d$, the model will predict $i$, the number of days until next order. So $d+i$ will give the precise date of a customer's next order. Variants are to define $i$ as the number of weeks or months until next order. The date will be less precise but the predictions might turn out to be more satisfying. Another possibility is to formalize it as a classification task with a binary outcome: \textit{"Will this customer make an order in the coming day/week/month ?"}. After discussion with Contoso, it has been decided to define the output of the model as \textit{the number of months until customer's next order}. In details, it has been decided first to work with regression models to have multiple ranges of outputs. Then, for the output's granularity (day, week or month), predicting the number of months until next order \textit{should} give stronger results. It will allow Contoso to asses the extent to which regression techniques and models are suited to this problem and, if the first phase is successful, try to improve the predictions with outputs at the week level.



\subsection{Predictions usage}
Based on the machine learning predictions, Contoso wants to reach customers before they make an order. The process is planned as follow:

\noindent\hspace*{0.8cm}  \texttt{1.} At the beginning of each month, compute the predictions for all active clients\footnote{A client is flagged as inactive if no more business can be made with him/her (in case of death or moving abroad for example)}. \\
\hspace*{0.8cm}           \texttt{2.} Fetch all clients with a prediction smaller than $2$ (order in current or coming month). \\
\hspace*{0.8cm}           \texttt{3.} Contact these clients only if they have not been contacted in the past two months.

The last step of this process is very important, as it will ensure that the company will not approach the same client twice about the same subject in a very short period.


\subsection{Machine learning scoring strategy}
Now that the machine learning goal and usage have been specified, the metrics for its success must be defined. As stated above, this is a regression model with a real number as output. Usual metrics for regression problems like \textit{Mean Absolute Error} (MAE) or \textit{Root Mean Squared Error} (RMSE) are not well suited for this project and the usage of predictions made by Contoso. As specified above, the company plans to get in touch with a client only if the prediction is below $2$. Therefore, if the output of a model is equal to $7.0$ or $4.35$, it ends up being the same for Contoso. This explains why MAE or RMSE cannot be used to evaluate machine learning models performance. 

As regression metrics are not applicable, models will be assessed with custom metrics inspired by classification tasks: precision, recall, and F1-score. 

To transform the regression problem into a classification one, model's predictions are classified into one of the following four classes: \texttt{0, 1, 2 or 3+}. Class \texttt{0} is meant for predictions between 0 and 1 (excluded) - clients that should make an order in the current month. Same idea for classes \texttt{1} and \texttt{2}. Class \texttt{3+} regroups all predictions with an output bigger or equal to 3 - the client's next order should occur in three or more months and will not be used by Contoso.

\begin{adjustwidth}{-1.7cm}{}
    \begin{minipage}[b]{0.60\linewidth}
    \resizebox{1.0\columnwidth}{!}{
        \begin{tabular}[t]{c|c|
                >{\columncolor[HTML]{EFEFEF}}c|c|
                >{\columncolor[HTML]{EFEFEF}}c }
                Customer & y\_true & y\_true class & y\_pred & y\_pred class \\ \hline
                A        & 2       & 2             & 3.32    & 3             \\
                B        & 0       & 0             & 0.4     & 0             \\
                C        & 0       & 0             & 0.1     & 0             \\
                D        & 5       & 3+            & 2.9     & 2            \\
                E        & 3       & 3+            & 0.8     & 0             \\
                F        & 1       & 1             & 1.6     & 1             \\
                G        & 1       & 1             & 0.8     & 0             \\
                H        & 2       & 2             & 1.4     & 1             \\
                I        & 9       & 3+            & 15.2    & 3+            \\
                J        & 0       & 0             & 2.5     & 2            
        \end{tabular}
        }
        \captionof{table}{Example of output truth and predictions}
        \label{table:example_truth_preds} 
    \end{minipage}
    \hspace{1.0cm}
    \begin{minipage}[b]{0.45\linewidth}
        \offinterlineskip
        \moveright 1cm \hbox{\raisebox{1.8cm}[10pt][10pt]{\rotatebox[origin=c]{90}{\parbox[c][0pt][c]{14cm}{True class\\[50pt]}}}\par}
        \hspace*{1cm}\MyHBoxT[\dimexpr5.1cm]{Predicted class}\vspace*{-0.4cm}
        \hspace*{1cm}\MyHBox{0}\MyHBox{1}\MyHBox{2}\MyHBox{3+}\vspace*{-0.2cm}
        \MyTBox{0}{2}{0}{1}{0}
        \MyTBox{1}{1}{1}{0}{0}
        \MyTBox{2}{0}{1}{0}{1}
        \MyTBox{3+}{1}{0}{1}{1}
        \captionof{table}{Confusion matrix for table \ref{table:example_truth_preds}}
        \label{table:example_confusion_matrix} 
    \end{minipage}
\end{adjustwidth}


Based on this classes assignment, a confusion matrix can be generated. Then, custom metrics inspired by precision, recall, and F1-score are computed:
\begin{itemize}
    \item \textbf{Precision}: Of all clients that make an order in the current month, how many were predicted with class 0 or 1? This metric computes the percentage of true orders caught by the model. 
    $$ Precision = \frac{t_0\_p_0 + t_0\_p_1}{t_0\_p_0 + t_0\_p_1 + t_0\_p_2 + t_0\_p_{3+}} $$
    
    For the example in table \ref{table:example_confusion_matrix}, precision is equal to $\frac{2+0}{2+0+1+0} = 0.667$
    
    \item \textbf{Recall}: From all clients that the model has predicted an order for the current month, how many did actually made an order in the current or coming month? The goal with this metric is to assert that the model is not always predicting 0, which will give a precision score of 100\%, but will make Contoso contact all clients. 
    $$ Recall = \frac{t_0\_p_0 + t_1\_p_0}{t_0\_p_0 + t_1\_p_0 + t_2\_p_0 + t_{3+}\_p_0} $$
    
    For the example in table \ref{table:example_confusion_matrix}, recall is equal to $\frac{2+1}{2+1+0+1} = 0.75$
    
    \item \textbf{F1-score}: Same as for classification tasks. The F1-score is a weighted average of the precision and recall.
    $$ \fscore = \frac{2*Precision*Recall}{Precision+Recall} $$

    For the example in table \ref{table:example_confusion_matrix}, the F1-score is equal to $\frac{2*0.667*0.75}{0.667+0.75} = 0.71$
\end{itemize}

In the formulas above, $t_i\_p_j$ corresponds to the sum of true orders occurring in $m+i$ months and predicted to occur in $m+j$ months, where $m$ is the current month. For example $t_0\_p_0$ corresponds to the sum of orders occurring in the current month and predicted as such (top left case in the confusion matrix).

The metric \textbf{F1-score} will be the benchmark metric to compare models. For all metrics, best score is \texttt{1.}0, worst is \texttt{0.0}.

The dataset used in this project is divided into training, testing and validation sets. The training set is composed of orders that took place between 2013 and 2016. It will be used to train all models. Data related to orders that occurred in 2017 form the testing set, use to test each model created. Finally, data associated with orders made from January 2018 to June 2018 are in the validation dataset, only used once to compute the final score in \ref{sec:best-model}.

% -------------------------------- Section: Data
\section{Data from CRM} \label{sec:crm-data}

\subsection{Data overview}
This section outlines the data gathering, cleaning, transformation, and feature engineering processes. Before doing some machine learning experiments, a first goal is to understand the data and understand the main factors that push a client to make an order. For the data analysis and machine learning experiments, the development has been done in \textit{Jupyter Notebooks} with a \texttt{Python 3} kernel, mainly relying on the \textit{Pandas} package.

Data comes from Contoso's CRM, a Dynamics 365 Online instance. By using the \textit{Web API} offered by Dynamics 365, all entities of the CRM have been retrieved and analyzed. From the 291 entities present in the CRM, only a few are useful for this project: \texttt{Orders}, \texttt{Account}, \texttt{Contact}, \texttt{Building}, and \texttt{Reservoir}. The relationships between those entities are straightforward: An order must be linked to an account and each account is linked to at least one contact, one of which must be the \textit{primary contact}. The \texttt{Contact} entity models a real person, where the \texttt{Account} entity models one of Contoso's customer, either a person or a company. Accounts can also be linked to one or more instance of the \texttt{Building} entity, which can also contain one or more \texttt{Reservoir}.

\begin{figure}[h]
    \centering
    \includegraphics[width=12cm]{images/entityDiagram.png}
    \caption[Entity diagram of the CRM data]{Entity diagram for Contoso CRM data}
    \label{fig:entity-diagram}
\end{figure}

\subsection{Orders}\label{sec:crm-orders}

The \texttt{Order} entity holds all information about an order made by an account. The CRM contains all orders received by Contoso since 2013 and for this project all orders considered ranged from January 2013 until June 2018, included. Once the raw data has been downloaded from all \texttt{Order} entities, the first steps are to clean this set of 808'532 orders. Indeed, even if a CRM saves its data in an organized manner and Dynamics 365 has some verification upon data entry, mistakes can still happen when a person enters data into the system. Some orders have been discarded based on their names -empty or not-, on their status -active or not- and on their internal characteristics (amount delivered equal to zero, delivery date in 1956, delivery date occurring before the order date, ..). This cleaning phase removes 28.79\% of the orders, which gives a data set of 575'679 orders, each order having 37 features.

All these orders have been made by 183'706 accounts, with a mean of 3.14 orders per account and a median of 2. Grouping accounts per the number of orders billed reveals that 50.77\% of accounts have made only one or two orders\footnote{Accounts which have made at least one order since 2013 - Figure \ref{fig-annex:orders_per_accounts} in the annexes.}. This demonstrates the singularity of a very competitive market in which clients often change their suppliers and fully benefit from the competition. In respect to the machine learning model, it will be very difficult to extract some common behavior and generalization due to the low amount of orders for those accounts. Therefore, accounts with less than three orders have been discarded, leading to a dataset composed of 451'986 orders (-21.49\%).

\begin{figure}[h]
    \centering
    \includegraphics[width=15cm]{images/order_month_year.png}
    \caption{Number of orders per month per year}
    \label{fig:order_per_monthyear}
\end{figure}

Because \textit{normal} accounts only order once per year, there is a seasonality effect visible when plotting the number of orders per month through  the years (figure \ref{fig:order_per_monthyear}). Even if the periodicity of orders isn't regular, a year experiences two pics of orders: one around March and another around October. On the opposite side, there have been fewer orders in the Spring, around June. The data indicate that the weather must play a role when accounts are making orders - it's colder in March and October than in June. As detailed in further sections \ref{sec:external-data}, the model will take some weather-related information into consideration.

As stated above, a typical customer will make an order once per year, with a number of months between two orders around 12 months, A mean of 12.67 months and a median of 11 months are retrieved by computing the average "wait" time between two orders for all accounts. Figure \ref{fig:orders-account-counts} shows that there are a lot of accounts waiting 11 months between two orders, but there are also accounts which order more frequently. On the other side, the number of accounts waiting 13 or more months between two orders decreases. Within this plot, there are two pics: the first logic one around 11 months and the second one around 23 months. This second pic is probably due to \textit{"jumper accounts"}, accounts that order one year with Contoso, the following year with the competition and, two years after their first orders, order again with Contoso. 

\begin{figure}[h]
    \centering
    \includegraphics[width=15cm]{images/accounts-average-time-orders.png}
    \caption[Average number of months between two orders]{Accounts grouped by the average number of months between two orders}
    \label{fig:orders-account-counts}
\end{figure}

This plot also reveals \textit{key-accounts}, accounts ordering at least once per month. Those accounts are most of the time companies, considered as very regular customers by Contoso. As detailed in section \ref{sec:data-shape-for-ml}, \textit{key-accounts} will not be a problem for the machine learning model.


\subsection{Accounts, Clients and Buildings}\label{sec:crm-accounts}
The \texttt{Account} entity holds information related to the clients of Contoso, either a natural or legal person. Due to imports from a legacy system, the Dynamics 365 instance of Contoso contains more than 1'100'000 accounts. Filtering those accounts based on the final orders from section \ref{sec:crm-orders} reduces the set to 88'552 accounts, with 244 features per account. The company stocks a lot of properties in their CRM for each account and the vast majority are null or not useful for the current task. After a review of all these features, only 11 will be kept, among which account's activeness and account's name, which will be compared to its primary client name.

Indeed, all accounts are linked to a primary contact. The \texttt{Contact} entity holds information about a physical person, like its address and phone number. From this entity, only two features are used: the birth date and the name. The birth date is used to check if the age of a person as an influence on it's buying habits. The contact name is compared to the account name. As mentioned above, the \texttt{Account} entity can be a natural or a legal person, but there is no feature in the CRM to specify this account's state. As this information is important -a legal person can make orders on a very frequent basis- a feature is built to capture this notion. An account's name can be compared to the name of its primary contact. If both names are equal, the account is considered as a natural person, otherwise the account is considered as a legal person, assuming that an account modeling a legal person will not have the same name as it's primary contact. As shown in figure \ref{fig:account-contact-name-orders}, accounts classified as a legal person (in red) made order much more frequently than "person-accounts" (in blue).

\begin{figure}[h]
    \centering
    \includegraphics[width=15cm]{images/account-contact-name-orders.png}
    \caption[Account and contact's name influence of order's frequency]{Number of months between two orders related to the account-contact name}
    \label{fig:account-contact-name-orders}
\end{figure}

Regarding the \texttt{Reservoir} entity, it holds information about the reservoir of a building (house, offices, cottage, ...). Even if not all accounts have such link (80.84\% of accounts are linked to at least one building), this entity will be used to build new features modeling the amount of product stocked and used, as detailed in section \ref{sec:ml-features}. The \texttt{Building} entity is only used to be the \textit{link} between accounts and reservoirs.


\subsection{External data}\label{sec:external-data}
In addition to the data coming from Dynamics 365, there are two important variables to take into account: price and weather. These two datasets are used to created new features, detailed in section \ref{sec:ml-features}.

As explained above, prices are mainly dependent on the stock market, which can change every day. It has been observed by Contoso that when the price is going down, more orders are coming in and when the price is going up, customers would rather hold, consume their stock and wait for a decrease in price to place a new order. A dataset containing the monthly prices of Contoso's product between 2013 and 2018 has been created. Unfortunately, it was not possible to compare this data with the monthly prices set by each competitor.

Weather data has also an influence on the ordering patterns of clients, as shown in section \ref{sec:crm-orders}. Weather data has been provided by \texttt{MeteoSwiss} with several daily metrics (weather, precipitations, wind, ...) associated to the 20 meteorological station across Switzerland. Based on the primary contact address, each \texttt{account} object has been linked to one of those stations and among all metrics, only the weather degrees \big[°C\big] have been used.


\section{Data for the machine learning}
The data for all CRM's entities need to be shaped so that the machine learning models can make use of it. This section details how the data from \ref{sec:crm-data} has been combined and then reshaped to be given to predictive models.

\subsection{Dataset construction}\label{sec:data-shape-for-ml}
As defined above, the model is planned to be used on a monthly basis and its output will be the number of months until customer's next order. For all accounts, one entry per month is created, from the month following client's first order until the month of last order and the output is the number of months until customer's next order. The data in these monthly entries contain features related to the account (shared among all entries), features related to the previous orders and features related to the current month, like the monthly weather for example. 

Figure \ref{fig:data-build-example} contains an example of the data creation process. Marked as yellow circles, this account has made four orders in April 2013, March 2014, January 2015 and March 2016. The first entry in the data relates to May 2013, the month following account's first order. The output of this entry is set to 11 months, the number of months between May 2013 and March 2014, date of account's second order. The same principle is applied to all months, until March 2016, date of the last order made by this account.

\begin{figure}[htbp]
    \hspace{-1cm}
    \includegraphics[width=17cm]{images/data-build-ml-example.png}
    \caption[Process to build data for machine learning]{Sketch of the process to build data for the machine learning models. Yellow circles model account's orders through time. The entry 1. corresponds to the month April 2013, the entry 17. to September 2014 and the entry 34. to February 2016.}
    \label{fig:data-build-example}
\end{figure}


% -------------------------------- Section: Customer Jounrey + Feature engineering
\subsection{Feature engineering}\label{sec:ml-features}

Before creating new features, all datasets must be combined together. Starting from right to left on figure \ref{fig:entity-diagram}, \texttt{Building} and \texttt{Reservoir} data are combined. If a \texttt{Building} is not linked to at least one \texttt{Reservoir}, it's discarded. Then the data is combined with the \texttt{Account} entity, a one-to-many relation (a given account can be linked to one or more buildings). From \texttt{Building} and \texttt{Reservoir}, only one feature is used: the volume of its reservoir. In the case of multiple buildings being linked to the same account, the average capacity of all reservoirs is considered.

Then, the entities \texttt{Account} and \texttt{Contact} are combined based on the account primary contact, with its address and birth date added to the account. This combination of the \texttt{Account}, \texttt{Contact}, \texttt{Building} and \texttt{Reservoir} entities builds a first set of features, defined as \textit{account-related} features.

Each row in the final machine learning data is composed of three types of features: the \textit{account-related} ones, features related to the current month and features about the previous order. Features related to the current month are mainly built from the two external data to encapsulate the variations of price and weather temperature through time. Finally, some characteristics of account previous order are used, like "\textit{Was it a big or a small order}?" or "\textit{When did this order occurred compare to the one before?}".


\begin{figure}[h]
    \centering
    \includegraphics[width=8cm]{images/data-build-ml-example-row.png}
    \caption[Features building for specific month]{Sketch of the features used for the row of September 2014 in figure \ref{fig:data-build-example}}
    \label{fig:data-build-row-example}
\end{figure}

The exhaustive list of the 122 features built can be found in annex \ref{annex:features-for-ml}. Those features are based on Contoso's knowledge of its business and analyses of \textit{Customer Journeys}. A \textit{Customer Journey} is a plot where all orders of a given account are visible, with the \texttt{Weather} and \texttt{Price} information through the months (see an example in figure \ref{fig-annex:customer-journey}). \textit{Customer Journeys} has been used as a starting point to understand the order's pattern of accounts, the influence of external data and the general behavior of Contoso's clients.

Some features have also been created after a trial and error phase while building and validating machine learning models. The property \texttt{feature\_importances\_} of \textit{Sklearn}'s models has also been used to rank features and build new ones based on it.



% -------------------------------- Section: Machine Learning Experimentation
\section{Machine Learning Experimentation} \label{sec:ml-experimentation}

It is the first time Contoso is involved in a machine learning project with the goal to model its customer's behavior. As such, all types of models and architectures are possible. In this project, it has been decided to start with several types of models and a simple set of features. The first goal is to evaluate the feasibility of the task and the family of machine learning models best suited for it. 


\subsection{Models experiments}
Python' package \textit{Scikit-learn} was used to build the models of this section. Scikit-learn enables to test testing different models very easily and several families of supervised learning models have been investigated, with the results reported in table \ref{tab:scores-simple-models}. Among the models investigated, there are \texttt{Liner models} (linear regression), \texttt{Tree models} (extra tree and decision tree), \texttt{Ensemble models} (random forest, extra trees, and bagging regressor) and \texttt{Neighbors models} (kneighbors regressor).

In this initial phase, all models have been trained with scikit-learn's default parameters and a subset of 17 features manually selected\footnote{Those features are marked with an * in annex \ref{annex:features-for-ml}.}. For all models created, the same seed state has been used.

\begin{table}[htb]
    \begin{tabular}{l|ccc|ccc}
                          & \multicolumn{3}{c|}{Train}                         & \multicolumn{3}{c}{Test}      \\
        \textbf{Model name}        & \textbf{Precision} & \textbf{Recall} & \textbf{F1-score} & \textbf{Precision} & \textbf{Recall} & \textbf{F1-score} \\ \hline
        Extra Tree        & 0.996     & 0.999  & 0.998                         & 0.324     & 0.389  & 0.354    \\
        Decision Tree     & 0.996     & 0.999  & 0.997                         & 0.267     & 0.394  & 0.318   \\
        KNeighbors        & 0.303     & 0.908  & 0.455                         & 0.131     & 0.508  & 0.208    \\
        Random Forest     & 0.696     & 0.998  & 0.820                         & 0.046     & 0.829  & 0.088    \\
        Bagging Regressor & 0.699     & 0.998  & 0.822                         & 0.045     & 0.829  & 0.084    \\
        ExtraTrees        & 0.996     & 0.999  & 0.998                         & 0.044     & 0.829  & 0.083    \\
        Linear Regression & 0.011     & 0.187  & 0.021                         & 0.028     & 0.305  & 0.051    
    \end{tabular}
    \caption{Prediction scores for first models, ranked by their F1-score on the testing set}
    \label{tab:scores-simple-models}
\end{table}

The first observation is that F1-scores aren't high and more than half of the models have a score below 0.10. It's also noticeable that all models are overfitting too much, except for the linear regression. The two tree models are giving the best predictions, with a good trade-off between precision and recall, where the ensemble models are very confident in their predictions (recall score above 80\%), but too cautious and catch less than 5\% of the orders.

The next experiments are related to the set of features used. With the predictions made by \texttt{Extra Tree} models, several features selection techniques have been used: Univariate selection where features are scored with the \textit{f\_regression}, Principal Component Analysis (PCA) and feature importance based on an \textit{ExtraTrees} model. The best F1-score turned out to be obtained with six features selected based on the \textit{ExtraTrees} model. With a score of \texttt{0.636}, the model trained with only six features achieves a score almost twice as good as the one obtained previously. It's apparent that this model gives it's better predictions when using a small number of features, as reported in figure \ref{fig:feature-selection}.

\begin{figure}[htbp]
    \centering
    \includegraphics[width=15cm]{images/feature_selection.png}
    \caption[F1-score based on feature selection]{F1-score obtain on the testing set with three different strategies for features selection.}
    \label{fig:feature-selection}
\end{figure}


After optimizing the set of features, \textit{ExtraTree} regressor's internals parameters can also be optimized. The primary goal is to reduce the overfitting produced by the tree. Two parameters have been fine-tuned: \texttt{max\_depth} and \texttt{min\_samples\_split}. The \texttt{max\_depth} parameter has no value by default, therefore the tree expands itself until all leaves are pure. The \texttt{min\_samples\_split} parameter controls the minimum number of samples required for a node to be split. By default, \textit{ExtraTree} will split each node which isn't a leaf and combined with no \texttt{max\_depth}, overfitting on the training data is obvious. 

Most relevant results of this fine-tuning phase are reported in table \ref{tab:tree-fine-tune}.

\begin{table}[htbp]
    \centering
    \begin{tabular}{c|c|c|c}
    \textbf{Max depth} & \textbf{Min samples split} & \textbf{Training F1-score} & \textbf{Testing F1-score} \\ \hline
    30	&  70   &  	0.738  &  0.715 \\
    50	&  100  &  	0.729  &  0.713 \\
    80	&  100  &  	0.729  &  0.713 \\
    30	&  100  &  	0.730  &  0.711 \\
    30	&  50   &  	0.748  &  0.710 \\
    50	&  70   &  	0.739  &  0.710 \\
    80	&  70   &  	0.739  &  0.710 \\
    50	&  50   &  	0.749  &  0.707 \\
    80	&  50   &  	0.749  &  0.707 \\
    30	&  10   &  	0.828  &  0.696 \\
    50	&  10   &  	0.854  &  0.683 \\
    80	&  10   &  	0.854  &  0.683 \\
    10	&  100  &  	0.449  &  0.466 \\
    10	&  10   &  	0.470  &  0.396 \\
    10	&  50   &  	0.439  &  0.388 \\
    10	&  70   &  	0.378  &  0.364 
    \end{tabular}
    \caption{F1-scores on training and testing data. Parameters investigated are the \texttt{max\_depth} and \texttt{min\_samples\_split}. Models trained on a set of six features. Table sorted by decreasing score on the testing set.}
    \label{tab:tree-fine-tune}
\end{table}

With the best combination of \texttt{max\_depth} and \texttt{min\_samples\_split}, the model reaches an F1-score of 0.715 on the testing data. It's also noticeable that the model is not overfitting anymore, with similar results on the training data. Regarding the two parameters investigates, \texttt{min\_samples\_split} has more impact on the predictions that \texttt{max\_depth}. Results show that the tree will not expand itself after 50 depth levels (same scores if the parameter is set to 50 or 80). Overfitting on the training data happens when the tree splits its nodes \textit{too early}, when it is composed of a small number of samples.


Alongside with experimenting models and new parameters, optimizing the data to train model has been examined. In previous experiments, data-related optimizations have only concern the features set, but there are some others possibilities. By investigating the data, three possible ways to reshape it were found. 
The first one consist of reducing the data by removing accounts with less than five entries. Then, for the remaining accounts, take only five random entries. The goal is to reduce the potential influence of accounts ordering very often and also reduce the potential noise in the data bring by accounts with very few entries, for example an account with four orders spread across five months. A second attempt to reduce the influence of \textit{extreme} accounts is to simply remove all accounts ordering every month or every two months. The third and last experiment to reshape the data is to have the same number of entries per output. All these three attempts to reshape the data never improved the F1-score previously obtained.

To summarize this section about experimenting with several models available in \textit{Scikit-learn}, the best model is an \textit{ExtraTree} regressor trained on a small set of features. After a fine-tuning phase, it achieves a F1-score of 0.715 on the testing dataset, a precision score of 0.813 and a recall score of 0.638.


\subsection{Neural Networks}
The \textit{ExtraTree} model has a big flaw: it's only using six features and none about the weather nor product's price. When adding that kind of features, the model isn't as good as before. Therefore, a new type of machine learning model is investigated: Neural Networks. Build with the \textit{Keras} framework, six different architectures of Neural Networks are experimented, all based on fully-connected layers. Similar to what has been done with previous models, several sets of manually selected features are used. The model achieving the best F1-score is a complete model composed of five fully-connected hidden layers trained with a set of 22 features and reaching an F1-score of \texttt{0.820} on the test data. Details results, and analysis of all tested models are reported in annex \ref{annex:nn-experiments}.

Once the model's architecture is chosen, internal parameters of the model can be investigated. Four parameters deserve some attention: the training \texttt{batch size}, model's compiler \texttt{loss function} and \texttt{optimizer} and the hidden layers \texttt{activation} function. Due to all possible values for those parameters, the fine-tuning phase has been done sequentially: starting from the best model in \ref{annex:nn-experiments}, optimize the training batch size, then the activation function of hidden layers, the model's compiler loss function and finally the model's optimizer function (results reported in annex \ref{annex:nn-fine-tuning}). Parameters of the model obtained are outlined in table \ref{tab:nn-final-parameters}.

\begin{table}[b]
    \centering
    \begin{tabular}{l|l|l}
        \textbf{Model part}           & \textbf{Parameter}                 & \textbf{Value}         \\ \hline
        Training                      & Number of epochs                   & 8                     \\
                                      & Input dimension                    & (X,22)                     \\ \hline
        Hidden layers                 & Number of neurons                  & 200-100-30                     \\
                                      & Activation function                & selu                     \\
                                      & Use bias                           & No                     \\ \hline
        \multicolumn{1}{l|}{Compiler} & \multicolumn{1}{l|}{Loss function} & \multicolumn{1}{l}{msle} \\
        \multicolumn{1}{l|}{}         & \multicolumn{1}{l|}{Optimizer}     & \multicolumn{1}{l}{adam}
    \end{tabular}
    \caption{Neural Network parameters for training final model}
    \label{tab:nn-final-parameters}
\end{table}


\subsection{LSTM}
A flaw present in previous models, neural networks included, is that the models are not account-specific. Of course, there are some account-specific features, but those are not enough to catch some use cases. Let's assume that an account has a specific ordering strategy: make a big order in June, then make a smaller order as soon as prices are going down, for example in September, and repeat the same ordering strategy every 15 months. For such account, the previous model will most probably not be able to catch its behavior with the features used. That's the main motivation for considering Long Short-Term Memory (LSTM) networks for this task. Being part of the Recurrent Neural Networks (RNN) family, LSTMs have the ability to recognize patterns in sequences of data. This is a different problem compared to other types of supervised learning since sequences impose an order on observations.

To be able to use LSTMs, the data need to be reshaped. Previous models were expecting a two-dimension input, but LSTMs requires data to be in a three-dimensions form: \texttt{[batch size, time steps, features size]}. To reshape the data, all entries related to the same account are sorted by month and year. Among all these entries, one is taken randomly and is considered as the last \textit{step} of the LSTM data, while all entries occurring before are considered as previous steps. Figure \ref{fig:lst-data-build} outlines this process. In order to have a dataset that a computer can handle, the LSTM data is composed of a maximum of three entries per account and each entry has a maximum of 15 steps (15 monthly data).

\begin{figure}[htbp]
    \centering
    \includegraphics[width=10cm]{images/lstm-data-build.png}
    \caption[LSTM data build]{Example of reshaping the data for LSTM models. From the initial data, the fourth entry has been selected randomly and is placed as the last step of the data.}
    \label{fig:lst-data-build}
\end{figure}

Similar to the experiments with previous neural networks, several model architectures have been investigated to train LSTMs: one with only the LSTM layer, others with some fully-connected dense layers after the LSTM part. With the first experiments, results were similar to the one obtained with only fully-connected layers but on the batch size fine-tuning phase, results didn't improve the testing F1-score. Results are available in annex \ref{annex:lstm-experiments}.



\subsection{Best model evaluation}
\label{sec:best-model}

The best model is a Neural Network composed of five fully connected hidden layers with 200, 150, 90, 60 and 30 neurons. Trained with the parameters reported in table \ref{tab:nn-final-parameters}, it achieves the following scores:

\noindent\hspace*{0.8cm}  1. \texttt{Training data}:   F1-score of \textbf{0.784}, precision of 0.741 and recall of 0.833 \\
\hspace*{0.8cm}           2. \texttt{Testing data}:    F1-score of \textbf{0.823}, precision of 0.804 and recall of 0.843 \\
\hspace*{0.8cm}           3. \texttt{Validation data}: F1-score of \textbf{0.820}, precision of 0.750 and recall of 0.906

Since the task is scored with metrics used for classification problem, the confusion matrix obtain with the validation data is generated in figure \ref{fig:cf-matrix-validation}.

\begin{figure}[htbp]
    \centering
    \includegraphics[width=15cm]{images/cf-matrix-validation_complete.png}
    \caption[Confusion matrix for validation data]{Confusion matrices for the validation data obtain with the best model. Two version: pure counts (left) and normalized per row (right).}
    \label{fig:cf-matrix-validation}
\end{figure}

When an account makes an order, almost one time on two the model predicts it at the correct month-level and 25\% of the cases the model predicts the order one month too late. Among the predictions made by the model, 62.94\% of the predictions are correct at the month-level while 27.63\% are wrong for one month. Regarding the errors, most of them are wrong for two months (6.89\%) and only 2.54\% of the predictions are completely wrong. This seems to indicate that the model makes robust predictions.

The wrong predictions of the model indicate that in 40\% of the cases the predictions is wrong of two months and wrong of three months in 22\%. This means that a slight improvement on the model can result in much more precise and trustworthy results. Error analysis also indicates that the model can sometimes be wrong for more than 40 months. This is due to \textit{inactive accounts}, clients used to order, never order more for quite some time and order again. By classifying an account as \textit{inactive} if it hasn't made an order in the last 22 months, 8.09\% of the errors would have been avoid.

Analyzing the \textit{customer journeys} of wrongly predicted accounts reveals that some orders are simply impossible to predict and doesn't follow a pattern. But it also shows that the model could be improved by giving more weight and importance to the previous order. This is particularly true for jumper accounts: the client made an infidelity one year between two orders with Contoso and the model is not able to realize that the client \textit{skipped} one order.

% -------------------------------- Section: Deployment
\section{Deployment} \label{sec:crm-deployment}
With the final machine learning model build and tested, its results must be integrated within the Dynamics 365 instance of Contoso. The end goal is to have for each account a prediction on the date of the next order. For this, two ways to make account predictions are defined:
\begin{itemize}
    \item Run the predictions every first Saturday for every month.
    \item Run the predictions for a specific account directly for the CRM interface, at anytime.
\end{itemize}

Both integration will be supported by \textit{Azure} services.

\subsection{Azure Architecture}
To deploy the model, four Azure services are used: \textit{Functions}, \textit{Queue Storage}, \textit{Web App} and \textit{Application Insights}. Azure Functions allow to run a piece of code on a serverless architecture and for the code to be scaled on demand. The Azure Queue Storage is a service to store large number of messages, which can latter be consumed via HTTP calls. Azure Web App creates web application with built-in autoscale and Azure Applications Insights are used to analyze performances, errors and logs messages. The connection between those services and Dynamics 365 are sketched in figure \ref{fig:azure-deployment}.

\begin{figure}[htbp]
    \centering
    \includegraphics[width=12cm]{images/azure-archi-weekly.png}
    \caption[Deployment architecture for scheduled predictions]{Connection between Azure services and Dynamics 365 to compute predictions on a recurring basis.}
    \label{fig:azure-deployment}
\end{figure}

The architecture used to integrate machine learning predictions inside Dynamics 365 contains the following steps:
\vspace*{-\baselineskip}
\begin{enumerate}[label=\texttt{\arabic*.}]
    \setcounter{enumi}{-1}
    \item The first Saturday of each month, at 03:00, the entire process is triggered.
    \item A first Azure Function retrieves all accounts of interest in Dynamics 365, via the Web API. Only active accounts with at least three orders are retrieved.
    \item Once all accounts are retrieved, they are added into a queue stored as an Azure Queue Storage.
    \item A second Azure Function is triggered each time new items are added into the Azure Queue Storage. This function will process five accounts per run.
    \item The Azure Function retrieves the predictions for the five accounts with an HTTP call to the Web App.
    \item The Web App holds the entire logic of the prediction. Build in Python, it makes use of the \textit{Flask} microframework to create HTTP endpoints. When the Web App receives the accounts from the Azure function, it will connect to the CRM to build the data for the machine learning. The ML model is stored in the Web App and used against the data just created. Predictions are then returned to the Azure Function.
    \item Once the Azure Function gets the predictions, it updates the CRM.
\end{enumerate}

One advantage of using this architecture is the connection between the Azure Queue Storage and the second Azure Function. It enables accounts to be consumed at a desired pace, in such a way to not overload the others components (Web App and Dynamics 365), to make the predictions before reaching the built-in timeout of Azure Functions and to have another queue containing failures. In case of an error during the process, accounts for which the predictions weren't computed are added to this queue, in such way that they do not stop the entire process. Through all steps, Azure Application Insights are used to store logs.

Experiments with this architecture showed a total running time of approximately 8 hours for 106'000 accounts. Compared to others services powered by Azure like the \textit{Azure Machine Learning}'s suite, this architecture is significantly less expensive, much more flexible and better suited for this use case.

The Azure Web App described above is also used to run the prediction for a specific account, upon CRM's user request. In that case, a JavaScript file stored in a Dynamics 365 \textit{Web resource} is sending an HTTP request to the Web App, which computes and returns the prediction.

\subsection{Integration with Dynamics 365}
The predictions made by the machine learning model must be visible and usable directly from Dynamics 365. Inside the \texttt{Account} entity page, a section \texttt{ML Predictions} is created with two fields: \texttt{Next Order Prediction} and \texttt{Last refresh}. An example of such page is visible in figure \ref{fig:dynamics-account-ml-screenshot}. The first field contains the predicted month for account's next order. The day the machine learning model is run is displayed below, in the second field. This enables CRM's users to see the prediction and when it was made. If the prediction is \textit{too old} or that new account-related information has been added, predictions can be refreshed on-demand by clicking the blue button \texttt{Run predictions}. This button is grayed out and inactive throughout the entire process (approx. 5-10 seconds). No refresh or page reload is necessary, both fields are automatically updated. If the process ends successfully, the button turns out green, but if a mistake happens, the button is red.

\begin{figure}[!htb]
    \centering
    \includegraphics[width=12cm]{images/dynamics-account-ml-screenshot.png}
    \caption[Dynamics 365 \textit{Account} entity page]{Dynamics 365 Account entity page. The part related to the machine learning prediction is in the blue rectangle. It contains two fields: \texttt{Next Order Prediction} and \texttt{Last refresh}, as well as a button \texttt{Run prediction}.}
    \label{fig:dynamics-account-ml-screenshot}
\end{figure}

As the prediction is stored as a CRM field, it can be used to create \textit{Marketing lists}. In Dynamics 365, creating a marketing list is usually the first step in a marketing campaign. For Contoso's use case, a marketing list can contain all accounts with the field \textit{Next Order Prediction} happening in the current or coming month. Then, those accounts can be contacted before they make an order and investigate competitor's offers.


% -------------------------------- Section: Further Work
\section{Discussion} \label{sec:use-case-further-work}
The models built for this task rely heavily on feature engineering. Indeed, data coming from CRM needs to be combined and reshaped to bring useful value to a machine learning model. The set of features used with the best model is not giving a lot of importance to the \textit{external} data: features related to the weather and price. Based on the customer journey visualization and Contoso's business knowledge, these two external factors must play a role in customer's ordering pattern. Therefore, new features or even new model can be investigated to incorporate such information.

A possible method to improve current models is to include features related to the customer's profile, for example customer's price sensitivity. Such features would help the model to weight the importance of price-related information. It will also enable Contoso to make personalized offers to customers. For example, if a customer is flagged as a \textit{jumper profile} with high price sensitivity, a personalized offer can be proposed in order to retain the customer.

A drawback of using a neural network model compared to a linear or tree-based model is that there are no justifications for the predictions. CRM users have access to the predictions, but there is no feedback nor explanation \textit{why} the neural network outputs such prediction.

The current architecture is built for the current Contoso's usage of predictions: monthly usage of predictions and no time constraint to make them. If the company decides to change the way of working with the predictions, to run them once or twice per week and build new models, the current architecture might not be adequate. The cloud-based components of Azure can be scaled, but creating data on-the-fly is a time-consuming step. Having part of the data static (account-related features) and only build the month-related features like the "\textit{number of weeks since last order}" or "\textit{weather of past week}" is an interesting option.

Finally, LSTM networks are an interesting option if Contoso's decided to change the scoring strategy. For example, if the number of weeks until next order are predicted instead of the number of months, the assumption is that LSTM will give better predictions than a fully-connected network. Indeed, the LSTM models built had a better $t_0\_p_0$ score compared to other models. The assumption is that LSTM models are more robust and confident regarding the \textit{closest} predictions.


% -------------------------------- Section: Conclusion
\section{Conclusion} \label{sec:use-case-conclusion}
In this project, real-world data coming from a CRM are used to create predictions regarding the date of the next customer order. Predictions are given by a neural network composed of five fully connected layers. Trained with all orders between 2013 and 2016, the model reaches an F1-score of 0.828. The F1-score used is a harmonic mean between the precision and recall of the model, two metrics computed with one-month tolerance on predictions.

An Azure-based architecture supports the incorporation of the predictions into the daily usage of Dynamics 365. This architecture enables recurrent predictions with user-intervention, as well as one-demand prediction requested directly by Dynamics 365 users.